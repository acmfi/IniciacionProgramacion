%%%%%%%%%%%%%%%%%%%%%%%%%%%%%%%%%%%%%%%%%%%%%%%%%%%%%%%
% MatPlotLib and Random Cheat Sheet
%
% Edited by Michelle Cristina de Sousa Baltazar
%
% http://matplotlib.org/api/pyplot_summary.html
% http://matplotlib.org/users/pyplot_tutorial.html
%
%%%%%%%%%%%%%%%%%%%%%%%%%%%%%%%%%%%%%%%%%%%%%%%%%%%%%%%

\documentclass{article}
\usepackage[landscape]{geometry}
\usepackage{url}
\usepackage{multicol}
\usepackage{amsmath}
\usepackage{amsfonts}
\usepackage{tikz}
\usetikzlibrary{decorations.pathmorphing}
\usepackage{amsmath,amssymb}

\usepackage{colortbl}
\usepackage{xcolor}
\usepackage{mathtools}
\usepackage{amsmath,amssymb}
\usepackage{enumitem}

\title{Cheatsheet - Introducción a la Programación}
\usepackage[brazilian]{babel}
\usepackage[utf8]{inputenc}

\advance\topmargin-.8in
\advance\textheight3in
\advance\textwidth3in
\advance\oddsidemargin-1.5in
\advance\evensidemargin-1.5in
\parindent0pt
\parskip2pt
\newcommand{\hr}{\centerline{\rule{3.5in}{1pt}}}
%\colorbox[HTML]{e4e4e4}{\makebox[\textwidth-2\fboxsep][l]{texto}
\begin{document}

\begin{center}{\huge{\textbf{Cheatsheet - Introducción a la Programación}}}\\
{\large ACM UPM}
\end{center}
\begin{multicols*}{3}

\tikzstyle{mybox} = [draw=black, fill=white, very thick,
    rectangle, rounded corners, inner sep=10pt, inner ysep=10pt]
\tikzstyle{fancytitle} =[fill=black, text=white, font=\bfseries]
%------------ CONTEÚDO CAIXA RANDOM ---------------
\begin{tikzpicture}
\node [mybox] (box){%
    \begin{minipage}{0.3\textwidth}
\begin{verbatim}
javac Mi_fichero.java
java  Mi_fichero
\end{verbatim}
    \end{minipage}
};
%------------ CAIXA RANDOM ---------------------
\node[fancytitle, right=10pt] at (box.north west) {Compilar y ejecutar};
\end{tikzpicture}


%------------ CONTEÚDO CAIXA MatPlotLib ---------------
\begin{tikzpicture}
\node [mybox] (box){%
  \begin{minipage}{0.3\textwidth}
    \begin{center}
      \begin{tabular}[h]{p{3cm} p{3cm}}
        \hline
        \emph{tipo}     & \emph{bits} \\ \hline
        \textbf{int}    & 32 \\
        \textbf{float}  & 32 \\
        \textbf{double} & 64 \\
        \textbf{char}   & 16 \\
        \textbf{Boolean}& \texttt{true | false} \\ \hline
      \end{tabular}
    \end{center}
  Declarar una variable (\emph{[ ] significa que puede estar o no}):

  \texttt{tipo nombre\_variable [= valor];}\\
  \texttt{double distancia; // Declara sin valor}\\
  \texttt{double distancia = 3.3;  // Declara con valor}\\
\end{minipage}
};
%------------ CAIXA PRELIMINARES ---------------------
\node[fancytitle, right=10pt] at (box.north west) {Tipos y variables};
\end{tikzpicture}
%------------ CONTEUDO EXEMPLO BASICO ---------------------
\begin{tikzpicture}
\node [mybox] (box){%
    \begin{minipage}{0.3\textwidth}
      \begin{center}
        \begin{tabular}[h]{l l l}
          \hline
          && Aritméticos \\ \hline
          +    & +=   & \texttt{variable}++\\
          -    & -=   & \texttt{variable}- -\\
          *    & *=   & \\
          /    & /=   & \\
          \%   & \%=  & \\ \hline
          && Lógicos \\\hline
          \&\& & $\parallel$ & \\ \hline
          && Relacionales \\\hline
          ==  & != $<=$ ~ $>=$  & \\
        \end{tabular}
      \end{center}
    \end{minipage}
};
%------------ EXEMPLO BASICO BOX ---------------------
\node[fancytitle, right=10pt] at (box.north west) {Operadores};
\end{tikzpicture}

\begin{tikzpicture}
  \node [mybox] (box){%
    \begin{minipage}{0.3\textwidth}

\begin{verbatim}
switch (variable_a_comparar) {
  case valor: instrucción; [beak;]
  default: ;
}
\end{verbatim}
\end{minipage}
};
\node[fancytitle, right=10pt] at (box.north west) {Control de flujo - Switch};
\end{tikzpicture}

\begin{tikzpicture}
  \node[] {
    \begin{minipage}{0.3\textwidth}
      \begin{center}
        \includegraphics[width=0.3\textwidth]{./img/acmblack.png}
        \hspace{15pt}
        \includegraphics[width=0.3\textwidth]{./img/qr.jpg}
      \end{center}

    \end{minipage}
  };
  \node[] at (1.6,-1.2) {Únete a ACM};
\end{tikzpicture}

%------------ CONTEUDO DOIS EIXOS ---------------------
\begin{tikzpicture}
\node [mybox] (box){%
    \begin{minipage}{0.3\textwidth}
  También conocido como ``control de flujo''. Permite cambiar \emph{qué} se ejecuta según ciertas \emph{condiciones}.

\begin{verbatim}
if (condición) {
  haz esto;
}

if (condición) {
  haz esto;
}else{ // Si no se cumple la condición
  haz lo otro;
}

if (condición){
  haz esto;
}else if(condición2){ //Si no, si...
  haz esto otro;
}
\end{verbatim}
  Los condicionales se pueden anidar \emph{ad infinitum}.
  Equivalencia en condicionales:

\begin{verbatim}
if (condición) {
  if (condición2){ }
}

if (condición && condición2) { }
\end{verbatim}
    \end{minipage}
};
%------------ DOIS EIXOS BOX ---------------------
\node[fancytitle, right=10pt] at (box.north west) {Control de flujo - Condicionales};
\end{tikzpicture}
%------------ CONTEÚDO COMANDOS DE TEXTO ---------------------
\begin{tikzpicture}
\node [mybox] (box){%
    \begin{minipage}{0.3\textwidth}
\begin{verbatim}
for ([declarar var];[cond];[op tras bloque]){}
while([condición]){}
\end{verbatim}

  Serían válidos por tanto:

\begin{verbatim}
for(;;){}
while(true){}
\end{verbatim}

  Correspondencia entre bucles:
\begin{verbatim}
for (int i=0;i<10;i++){
  System.out.println(i);
}

int i=0;
while(i<10){ System.out.println(i); }
\end{verbatim}
    \end{minipage}
};
%------------ COMANDOS DE TEXTO BOX ---------------------
\node[fancytitle, right=10pt] at (box.north west) {Bucles - for y while};
\end{tikzpicture}
%------------ CONTEUDO PROPRIEDADES ---------------------
\begin{tikzpicture}
\node [mybox] (box){%
    \begin{minipage}{0.3\textwidth}
\begin{verbatim}
(tipo|void) nombre ([argumentos]){}
int duplicar (int x) { return x*2; }
void no_devuelve () { System.out.println("<3"); }


class Clase {
  public static void main (String[] args) (
    //Código a ejecutar al llamarse con java
  }
}

\end{verbatim}
  La función puede tener \emph{modificadores} como: \texttt{public, private, static}. Esto queda fuera del temario de Prog I.

    \end{minipage}
};
%------------ PROPRIEDADES BOX ----------------
\node[fancytitle, right=10pt] at (box.north west) {Funciones};
\end{tikzpicture}

\begin{tikzpicture}
\node [mybox] (box){%
  \begin{minipage}{0.3\textwidth}
    \begin{center}
      \begin{tabular}{|cc|cc|cc|}
        \hline
        13          & CR         & 71         & G            & 100 & d               \\ \hline
        32          &            & 72         & H            & 101 & e               \\ \hline
        40          & (          & 73         & I            & 102 & f               \\ \hline
        41          & )          & 74         & J            & 103 & g               \\ \hline
        42          & *          & 75         & K            & 104 & h               \\ \hline
        43          & +          & 76         & L            & 105 & i               \\ \hline
        44          & ,          & 77         & M            & 106 & j               \\ \hline
        \textbf{48} & \textbf{0} & 78         & N            & 107 & k               \\ \hline
        49          & 1          & 79         & O            & 108 & l               \\ \hline
        50          & 2          & 80         & P            & 109 & m               \\ \hline
        51          & 3          & 81         & Q            & 110 & n               \\ \hline
        52          & 4          & 82         & R            & 111 & o               \\ \hline
        53          & 5          & 83         & S            & 112 & p               \\ \hline
        54          & 6          & 84         & T            & 113 & q               \\ \hline
        55          & 7          & 85         & U            & 114 & r               \\ \hline
        56          & 8          & 86         & V            & 115 & s               \\ \hline
        57          & 9          & 87         & W            & 116 & t               \\ \hline
        \textbf{65} & \textbf{A} & 88         & X            & 117 & u               \\ \hline
        66          & B          & 89         & Y            & 118 & v               \\ \hline
        67          & C          & 90         & Z            & 119 & w               \\ \hline
        68          & D          & \textbf{97}& \textbf{a}   & 120 & x               \\ \hline
        69          & E          & 98         & b            & 121 & y               \\ \hline
        70          & F          & 99         & c            & 122 & z               \\ \hline
      \end{tabular}
    \end{center}
  \end{minipage}
};
\node[fancytitle, right=10pt] at (box.north west) {Anexo};
\end{tikzpicture}

\vspace{15pt}{\tiny ** La asociación no se hace responsable del
  mal uso de este documento.}
\end{multicols*}
\end{document}