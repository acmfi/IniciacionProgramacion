\documentclass[10pt,twocolumn]{article}

\usepackage[spanish]{babel}
\usepackage[utf8]{inputenc}
\usepackage{geometry}

\geometry{
  a4paper,
  left=10mm,
  top=10mm,
  right=4mm 
  }
\title{Cheatsheet - Iniciación a la Programación}
\author{ACM UPM}

\begin{document}
\maketitle

\section{Tipos}
\label{sec:tipos}

\begin{table}[h!]
  \centering
  \begin{tabular}[h]{p{3cm} p{3cm}}
    \hline
    \emph{tipo}     & \emph{bits} \\ \hline
    \textbf{int}    & 32 \\
    \textbf{float}  & 32 \\
    \textbf{double} & 64 \\
    \textbf{char}   & 16 \\
    \textbf{Boolean}& \texttt{true | false} \\ \hline
  \end{tabular}
\end{table}

Declarar una variable (\emph{[ ] significa que puede estar o no}):

\begin{verbatim}
tipo nombre_variable [= valor];

double distancia; // Declara pero no tiene valor
double distancia = 3.3; // Declara y tiene valor
\end{verbatim}

\section{Operadores}
\label{sec:operadores}

\begin{table}[h]
  \begin{tabular}[h]{l l l}
    \hline
    && Aritméticos \\ \hline
    +    & +=   & \texttt{variable}++\\ 
    -    & -=   & \texttt{variable}- -\\
    **   & *=   & \\
    /    & /=   & \\
    \%   & \%=  & \\ \hline
    && Lógicos \\\hline
    \&\& & $\parallel$ & \\ \hline
    && Relacionales \\\hline
     ==  & != $<=g$ ~ $>=$  & \\
  \end{tabular}
\end{table}

\section{Condicionales}
\label{sec:condicionales}

También conocido como ``control de flujo''. Permite cambiar \emph{qué} se ejecuta según ciertas \emph{condiciones}.

\begin{verbatim}
if (condición) {
  haz esto;
}

if (condición) {
  haz esto;
}else{ // Si no se cumple la condición
  haz lo otro;
}

if (condición){
  haz esto;
}else if(condición2){ //Si no, si...
  haz esto otro;
}
\end{verbatim}
Los condicionales se pueden anidar \emph{ad infinitum}.
Equivalencia en condicionales:

\begin{verbatim}
if (condición) {
  if (condición2){ }
}

if (condición && condición2) { }
\end{verbatim}

\section{Bucles}
\label{sec:bucles}

\begin{verbatim}
for ([declarar var];[cond];[op tras bloque]){}
while([condición]){}
\end{verbatim}

Serían válidos por tanto:

\begin{verbatim}
for(;;){}
while(true){}
\end{verbatim}

Correspondencia entre bucles:
\begin{verbatim}
for (int i=0;i<10;i++){
  System.out.println(i);
}

int i=0;
while(i<10){
  System.out.println(i);
}
\end{verbatim}

\section{Funciones}
\label{sec:funciones}

\begin{verbatim}
(tipo|void) nombre ([argumentos]){}
int duplicar (int x) { return x*2; }
void no_devuelve () { System.out.println("<3"); }
\end{verbatim}
La función puede tener \emph{modificadores} como: \texttt{public, private, static}. Esto queda fuera del temario de Prog I.

\section{Anexo}
\label{sec:anexo}

\begin{table}[h!]
  \centering
  \begin{tabular}{|cc|cc|cc|}
  \hline
  13          & CR         & 71         & G            & 100 & d               \\ \hline  
  32          &            & 72         & H            & 101 & e               \\ \hline
  40          & (          & 73         & I            & 102 & f               \\ \hline
  41          & )          & 74         & J            & 103 & g               \\ \hline
  42          & *          & 75         & K            & 104 & h               \\ \hline
  43          & +          & 76         & L            & 105 & i               \\ \hline
  44          & ,          & 77         & M            & 106 & j               \\ \hline
  \textbf{48} & \textbf{0} & 78         & N            & 107 & k               \\ \hline
  49          & 1          & 79         & O            & 108 & l               \\ \hline
  50          & 2          & 80         & P            & 109 & m               \\ \hline
  51          & 3          & 81         & Q            & 110 & n               \\ \hline
  52          & 4          & 82         & R            & 111 & o               \\ \hline
  53          & 5          & 83         & S            & 112 & p               \\ \hline
  54          & 6          & 84         & T            & 113 & q               \\ \hline
  55          & 7          & 85         & U            & 114 & r               \\ \hline
  56          & 8          & 86         & V            & 115 & s               \\ \hline
  57          & 9          & 87         & W            & 116 & t               \\ \hline
  \textbf{65} & \textbf{A} & 88         & X            & 117 & u               \\ \hline
  66          & B          & 89         & Y            & 118 & v               \\ \hline
  67          & C          & 90         & Z            & 119 & w               \\ \hline
  68          & D          & \textbf{97}& \textbf{a}   & 120 & x               \\ \hline
  69          & E          & 98         & b            & 121 & y               \\ \hline
  70          & F          & 99         & c            & 122 & z               \\ \hline
  \end{tabular}
  \caption{Mini-tabla ASCII }
  \label{tab:ascii}
\end{table}

\end{document}
